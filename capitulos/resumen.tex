\chapter*{Resumen}

En este proyecto se estudia el estado actual de la aplicabilidad de las tecnologías de tipo \textit{blockchain} al Aprendizaje Federado (FL) para la mitigación de ataques adversarios. Para ello se implementan arquitecturas estándar de FL y de \textit{blockchain} aplicado a FL, comprobándose su capacidad como defensa ante ataques y se propone una nueva arquitectura que mejora las anteriores en este sentido.

Para introducir el proyecto y conocer mejor sus fundamentos, se realiza una introducción al álgebra lineal, teoría de la probabilidad, inferencia y teoría de la información. De forma seguida, se explica la base de la optimización de funciones no lineales, el operador de agregación Krum y nociones sobre Aprendizaje Automático, más concretamente Aprendizaje Profundo.

Los siguientes capítulos se enfocan en explicar FL, así como los distintos ataques a los que es vulnerable este esquema, haciendo un especial énfasis en los tratados en el trabajo. También se explica la tecnología \textit{blockchain} y sus beneficios en sistemas distribuidos.

Finalmente, se explica el estado actual del paradigma de \textit{blockchain} aplicado a FL, se expone la hipótesis de cómo ciertas arquitecturas podrían servir como una capa de defensa para FL y se propone una nueva arquitectura enfocada en la seguridad del modelo. Esta nueva arquitectura, Krum Federated Chain (KFC), se basa en las propuestas actuales de aplicabilidad de \textit{blockchain} a FL con un efoque especial en defensa logrado por la aplicación del operador de agregación Krum. Esto es seguido por la realización y análisis de una serie de experimentos para contrastar la hipótesis planteada y la propuesta, observando como KFC mejora los métodos de defensa ofrecidos por \textit{blockchain} mantiendo el modelo seguro bajo escenarios más restrictivos, junto a conclusiones y vías de trabajo futuro.


\textbf{Palabras clave}: aprendizaje federado, ataques adversarios, redes neuronales, blockchain, clasificación de imágenes, aprendizaje automático, optimización no lineal.